% !TEX root = ../../thesis.tex

\section{Other Tips}%
\label{sec:basics-other}%

\subsection{Dummy Text / Lipsum}%
If you're blocking out structure and want some filler text, you can use the \verb|\lipsum| command.
Make sure to look up the package documentation on ctan\footnote{\url{https://www.ctan.org/pkg/lipsum}} as it's very flexible.

\lipsum[4]


\subsection{Code Snippets}%
This template uses the \verb|listings| package for inserting code snippets both inline and as blocks/figures.
Again, check the documentation on ctan, which I recommend any time you have questions about a package.

\begin{lstlisting}[language=c++, numbers=left, caption={Checking if a number is even using bit magic.}]
bool isEven( const unsigned int uValue ) {
  return (uValue & (uValue - 1)) == 0;
}
\end{lstlisting}

An alternative package to try is \verb|minted| which you may prefer over \verb|listings|.
It is not included in this template, so you will have to include it manually.


\subsection{Local Environment}%
I do not recommend using Overleaf for your thesis, but instead recommend setting up a local development environment.
This is because you are likely to hit the compilation time limit on Overleaf at some point late in your thesis, and switching environments last second could be a significant disturbance.
Moreover, in my own thesis I've used several advanced features or packages that outright crash Overleaf.
On the flip side, I do recommend having a blank version of this template on Overleaf so that you can quickly play around with formatting (such as building a complex table) and get instant feedback on errors and visuals.

To get \LaTeX{} on your machine, you can visit the main \LaTeX{} download page at \url{https://www.latex-project.org/get} and choose your operating system appropriately.
Personally, I use MiKTeX\footnote{\url{https://miktex.org}} on Windows and MacTeX\footnote{\url{https://www.tug.org/mactex}} on Mac.

To actually write \LaTeX{}, I use \textit{Visual Studio Code} with several helpful plugins.
I recommend using \textit{Better Comments} (colored comments to help make notes easier to find), \textit{LaTeX Workshop} (for mature LaTeX support), and \textit{Todo Tree} (to make ``TODO:'' comments stand out and findable in a list).


\subsection{Version Control}%
I recommend that you store your thesis in some kind of version control such as git.
You will inevitably want to go back and see previous versions of your document, and this is a great way of doing so.

You may have noticed looking through these documents that I have a \textbf{single} sentence per line.
This is intentional.
If you have a gigantic paragraph per line and you change even a single character, the entire line becomes modified, which makes detecting what changed quite hard.
If you instead have a line per sentence, then this makes it easier to find changes as they happen.

\subsection{Nomenclature}
Nomenclature lets you define a glossary of terms at the start of your document.
See the \LaTeX{} source here for how to define them.
Note that seeing changes to nomenclature requires building an index.
See \S\ref{ch:compiling} for more details.

% See the `nomencl' package documentation for details (https://www.ctan.org/pkg/nomencl).
\nomenclature[z-ALU]{ALU}{Arithmetic Logic Unit}
\nomenclature[z-BEM]{BEM}{Boundary Element Method}
\nomenclature[z-CFD]{CFD}{Computational Fluid Dynamics}
\nomenclature[z-CK]{CK}{Carman - Kozeny}
\nomenclature[z-DEM]{DEM}{Discrete Element Method}
\nomenclature[z-FEM]{FEM}{Finite Element Method}
\nomenclature[z-FLOP]{FLOP}{Floating Point Operations}
\nomenclature[z-FPU]{FPU}{Floating Point Unit}
\nomenclature[z-FVM]{FVM}{Finite Volume Method}
\nomenclature[z-GPU]{GPU}{Graphics Processing Unit}
\nomenclature[z-LBM]{LBM}{Lattice Boltzmann Method}
\nomenclature[z-LES]{LES}{Large Eddy Simulation}
\nomenclature[z-MPM]{MPM}{Material Point Method}
\nomenclature[z-MRT]{MRT}{Multi-Relaxation Time}
\nomenclature[z-PCI]{PCI}{Peripheral Component Interconnect}
\nomenclature[z-PFEM]{PFEM}{Particle Finite Element Method}
\nomenclature[z-RVE]{RVE}{Representative Elemental Volume}
\nomenclature[z-SH]{SH}{Savage Hutter}
\nomenclature[z-SM]{SM}{Streaming Multiprocessors}

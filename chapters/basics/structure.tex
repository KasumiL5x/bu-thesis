% !TEX root = ../../thesis.tex

\section{Structure}%
\label{sec:basics-structure}%

\LaTeX{} can organize and number your document automatically with several sectioning commands.
From the highest to the lowest, the commands are:

\begin{itemize}[noitemsep,left=1pt]
  \item \verb|\part{name}|
  \item \verb|\chapter{name}|
  \item \verb|\section{name}|
  \item \verb|\subsection{name}|
  \item \verb|\subsubsection{name}|
  \item \verb|\paragraph{name}|
  \item \verb|\subparagraph{name}|
\end{itemize}

\noindent In your thesis, you are likely to only use \verb|\chapter|, \verb|\section|, \verb|\subsection|, and \verb|\subsubsection|.
The file \verb|basics.tex| defines a chapter, which is represented above by ``Chapter 2''.
This file, \verb|structure.tex|, defines a section, which is represented above as ``2.1''.
When we add subsections, they would be represented as ``2.1.1'' and so on.

The style of these sectioning commands has been customized for this thesis. Look for ``Custom Titles'' in \verb|preamble.tex| for more information.

\subsection*{Unnumbered Sectioning}%
This subsection is not listed in the table of contents and doesn't have a number.
You can achieve this with any of the sectioning commands by appending them with an asterisk.

% !TEX root = ../../thesis.tex

% I recommend having a separate 'main file' for each chapter.
% In this case, the 'Introduction' chapter lives in [chapters/introduction/introduction.tex].
\chapter{Introduction}%

% This command sets a name that we can later refer to (e.g., See Chapter~\ref{ch:introduction}).
% It's a good idea to have them, but they aren't required.
% While not required at all, here are my own personal rules for naming labels:
% All label names are in the format `category:name`.
% The `name` portion should be a hierarchy from the chapter down through sections if sections are labeled too to the element.
%  They should be separated by hyphens to allow for underscores in individual names. (e.g., ch-sec-fig_name).
% The `category` portion should be a category as follows:
% 	 Chapter: ch
% 	 Section: sec
% Subsection: subsec
% 	Appendix: appdx
% 	   Table: table
% 	  Figure: fig
\label{ch:introduction}%

% This sets the root folder for your graphics for THIS chapter. In this case, [chapters/introduction/figures].
% It stops bloating of a single 'figures' folder and helps to separate them out.
\graphicspath{{chapters/introduction/figures/}}

Welcome! Here is a brief description of what \LaTeX{} is.

\LaTeX{} is a document preparation system for the \TeX{} typesetting program.
It offers programmable desktop publishing features and extensive facilities for automating most aspects of typesetting and desktop publishing, including numbering and cross-referencing, tables and figures, page layout, bibliographies, and much more.
\LaTeX{} was originally written in 1984 by Leslie Lamport and has become the dominant method for using \TeX{}; few people write in plain \TeX{} anymore.
The current version is \LaTeXe{}.

The remainder of this document will briefly expose you to the basics of \LaTeX{}.

% ****************************** Packages, Commands, Settings ******************************

% ****************************** Custom Margin ******************************
% Use the document class option 'custommargin' to use this section.
% Set {innerside, outerside, top, bottom} and other page dimensions.
\ifsetCustomMargin
  \RequirePackage[left=40mm,right=25mm,top=35mm,bottom=30mm]{geometry}
  \setFancyHdr % To apply fancy header after geometry package is loaded.
\fi




% ****************************** Paragaph Spacing ******************************
% Add spaces between paragaphs.
% \setlength{\parskip}{0.5em}

% Enabling ragged bottom avoids extra whitespaces between paragraphs.
\raggedbottom

% Use this to remove excess top spacing for enumerations, lists, and descriptions.
% \usepackage{enumitem}
% \setlist[enumerate,itemize,description]{topsep=0em}




% ****************************** Line Spacing ******************************
% Pick the linespacing that suits your university guidelines.  Default is one-half line spacing.
% \doublespacing
% \onehalfspacing
% \singlespacing




% ****************************** Custom Fonts ******************************
% Use the document class option 'customfont' to use this section.  If you don't,a default LaTeX font will be used.
\ifsetCustomFont
  % For example:
  % 
  % \RequirePackage{helvet}
  % For use with XeLaTeX.
  % \setmainfont[
  %   Path           = ./libertine/opentype/,
  %   Extension      = .otf,
  %   UprightFont    = LinLibertine_R,
  %   BoldFont       = LinLibertine_RZ, % Linux Libertine O Regular Semibold
  %   ItalicFont     = LinLibertine_RI,
  %   BoldItalicFont = LinLibertine_RZI, % Linux Libertine O Regular Semibold Italic
  % ]
  % {libertine}
  % Load font from system font.
  % \newfontfamily\libertinesystemfont{Linux Libertine O}
\fi




% ****************************** Custom Packages ******************************
% Here you can add your own custom packages and configure them.  The ones listed are common suggestions.

% \usepackage{algpseudocode} % Algorithms and pseudocode (https://ctan.org/pkg/algorithmicx).

\RequirePackage[labelsep=space,tableposition=top]{caption} % Customize float captions (https://ctan.org/pkg/caption).
\renewcommand{\figurename}{Fig.} % To support older versions of captions.sty.

% \usepackage{rotating} % Rotation of figures and tables (https://ctan.org/pkg/rotating).
% \usepackage{wrapfig} % Wrap text around figures and tables (https://ctan.org/pkg/wrapfig).
\usepackage{subcaption} % Multiple captions within a figure (https://ctan.org/pkg/subcaption).
\usepackage{booktabs} % Professional looking tables (https://ctan.org/pkg/booktabs).
\usepackage{multirow} % Table cells that can span multiple rows (https://ctan.org/pkg/multirow).
% \usepackage{multicol} % Table cells that can span multiple columns (https://ctan.org/pkg/multicol).
% \usepackage{longtable} % Tables that can spread over multiple pages (https://ctan.org/pkg/longtable).
% \usepackage{tabularx} % Extends tabular to include paragraph-like column expansion; tabulary also exists (https://ctan.org/pkg/tabularx).
% \usepackage{float} % Makes placing figures and tables easier.  In particular, [H] for exact placement (https://ctan.org/pkg/float).
% \restylefloat{figure} % If you use the float package, enable this.

\usepackage{siunitx} % Ability to use SI unit symbols (https://ctan.org/pkg/siunitx).

% \usepackage[perpage]{footmisc} % Customizable footnote options (https://ctan.org/pkg/footmisc).




% ****************************** Enumeration Breaking ******************************
% Set boundaries to stop enumations (etc.) from breaking across pages in an ugly way (defaults to 10000).
% \clubpenalty=500
% \widowpenalty=500




% ****************************** References ******************************
% Packages and settings relating to referencing.

% \usepackage{cleveref} % Automatic referencing without explicitly stating fig/table (https://ctan.org/pkg/cleveref).

% Use the document class option 'custombib' to use this section.
\ifuseCustomBib
   \RequirePackage[square, sort, numbers, authoryear]{natbib} % Provide your custom bib package here.  In this case, natbib.
% If you would like to use biblatex for your reference management, as opposed to the default `natbibpackage` pass the option `custombib` in the document class. Comment out the previous line to make sure you don't load the natbib package. Uncomment the following lines and specify the location of references.bib file.
% \RequirePackage[backend=biber, style=numeric-comp, citestyle=numeric, sorting=nty, natbib=true]{biblatex}
% \bibliography{references/references} %Location of references.bib only for biblatex
\fi

% Changes the default 'Bibliography' name into 'References'.
\renewcommand{\bibname}{References}




% ****************************** Custom Commands ******************************
% Custom commands can be defined and configured here.

% For changing the name of ToC, LoF, LoT.
% \renewcommand{\contentsname}{My Table of Contents}
% \renewcommand{\listfigurename}{My List of Figures}
% \renewcommand{\listtablename}{My List of Tables}

% ToC depth and section numbering depth.
\setcounter{secnumdepth}{2}
\setcounter{tocdepth}{2}

% Changes the name of the nomenclature section.
% \renewcommand{\nomname}{Symbols}

% The default value of \appendixtocname and \appendixpagename is 'Appendices'.  These commands can change that.
% \renewcommand{\appendixtocname}{List of appendices}
% \renewcommand{\appendixname}{Appndx}




% ****************************** Draft Mode Settings ******************************
% Sets up options for draft mode (enable by using the draft option).

% Uncomment to disable figures in draft mode.
% \setkeys{Gin}{draft=true}  % Set draft to false to enable figures in draft mode.

% Set the watermark text.
% \SetDraftText{DRAFT}

% Set the watermark location.  Can use 'top' or 'bottom'.
% \SetDraftWMPosition{bottom}

% Set the version number for the draft.  Defaults to v1.0.
% \SetDraftVersion{v1.1}

% Draft text grayscale value (where 0 is black and 1 is white).  Defaults to 0.75.
% \SetDraftGrayScale{0.8}




% ****************************** Todo Notes Settings ******************************
% Uncomment the following lines to enable todonotes.
% \ifsetDraft
% 	\usepackage[colorinlistoftodos]{todonotes}
% 	\newcommand{\mynote}[1]{\todo[author=kks32,size=\small,inline,color=green!40]{#1}}
% \else
% 	\newcommand{\mynote}[1]{}
% 	\newcommand{\listoftodos}{}
% \fi

% Example todo: \mynote{Enter your note text here.}

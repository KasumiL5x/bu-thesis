% !TEX root = ./thesis.tex

% ****************************** Intentionally blank pages ******************************
% If you're using double-sided mode, chapters will start on a specific side possibly leaving a page blank.
% This package adds text to that blank page to let the reader know it's not an error.
% https://gist.github.com/philipptempel/5220000
% https://www.tug.org/TUGboat/tb31-3/tb99merciadri.pdf
\makeatletter
    \def\cleardoublepage{\clearpage%
        \if@twoside%
            \ifodd\c@page\else
                \vspace*{\fill}
                \hfill
                \begin{center}
                This page is intentionally left blank.
                \end{center}
                \vspace{\fill}
                \thispagestyle{empty}
                \newpage
                \if@twocolumn\hbox{}\newpage\fi
            \fi
        \fi
    }
\makeatother



% ****************************** Smaller footnote URLs ******************************
% Note that the style file sets its own custom value for this for references, so this will override it fully.
% Not recommended to enable by default, but it's here if you need it.
% \urlstyle{same}
% \renewcommand{\UrlFont}{\footnotesize\ttfamily}



% ****************************** Custom Margin ******************************
% Use the document class option 'custommargin' to use this section.
% Set {innerside, outerside, top, bottom} and other page dimensions.
% Bournemouth standard is >=40mm for the bound edge and >=20mm for the rest.
% If you use double sided printing, set the right to >=40mm too.
\ifsetCustomMargin%
  \RequirePackage[left=40mm,right=20mm,top=20mm,bottom=20mm]{geometry} % originally 40/25/35/30
  \setFancyHdr% To apply fancy header after geometry package is loaded.
\fi



% ****************************** Paragraph Spacing ******************************
% Add spaces between paragraphs.
% \setlength{\parskip}{0.5em}

% Enabling ragged bottom avoids extra whitespaces between paragraphs.
\raggedbottom%

% Use this to remove excess top spacing for enumerations, lists, and descriptions.
\usepackage{enumitem}
% \setlist[enumerate,itemize,description]{topsep=0em}



% ****************************** Line Spacing ******************************
% Pick the linespacing that suits your university guidelines. Default is one-half line spacing.
% \singlespacing%
\onehalfspacing%
% \doublespacing%



% ****************************** Custom Fonts ******************************
% Use the document class option 'customfont' to use this section. If you don't,a default LaTeX font will be used.
\ifsetCustomFont%
  % This will use the font from  acmart.cls (JOCCH etc.). It's a nice standard publication font.
  \RequirePackage[tt=false, type1=true]{libertine}
  \RequirePackage[libertine]{newtxmath}

  % For example:
  %
  % \RequirePackage{helvet}
  % \renewcommand{\familydefault}{\sfdefault}
  % \setmainfont{cmss}
  % For use with XeLaTeX.
  % \setmainfont[
  %   Path           = ./libertine/opentype/,
  %   Extension      = .otf,
  %   UprightFont    = LinLibertine_R,
  %   BoldFont       = LinLibertine_RZ, % Linux Libertine O Regular Semibold
  %   ItalicFont     = LinLibertine_RI,
  %   BoldItalicFont = LinLibertine_RZI, % Linux Libertine O Regular Semibold Italic
  % ]
  % {libertine}
  % Load font from system font.
  % \newfontfamily\libertinesystemfont{Linux Libertine O}
\fi



% ****************************** Footnotes ******************************
% \usepackage[perpage]{footmisc} % Customizable footnote options (https://ctan.org/pkg/footmisc).

% Replaces the standard \footnote{} command and places all footnotes at the END of a document, like in some books.
% If you want footnotes to appear inline at the bottom of the page they are defined on, simply comment this out and rebuild.
\usepackage{enotez}
\setenotez{backref=true,totoc=chapter,split=chapter}
\let\footnote=\endnote%



% ****************************** Custom Packages ******************************
% Here you can add your own custom packages and configure them. The ones listed are common suggestions.

% \usepackage{algpseudocode} % Algorithms and pseudocode (https://ctan.org/pkg/algorithmicx).

\RequirePackage[labelsep=space,tableposition=top]{caption} % Customize float captions (https://ctan.org/pkg/caption).
\renewcommand{\figurename}{Fig.} % To support older versions of captions.sty.

% Packages related to figures or tables.
\usepackage{subfig} % Multiple figures (supersedes the subfigure/subcaption package).
\usepackage{booktabs} % Professional looking tables (https://ctan.org/pkg/booktabs).
\usepackage{multirow} % Table cells that can span multiple rows (https://ctan.org/pkg/multirow).
\usepackage{multicol} % Table cells that can span multiple columns (https://ctan.org/pkg/multicol).
% \usepackage{longtable} % Tables that can spread over multiple pages (https://ctan.org/pkg/longtable).
% \usepackage{tabularx} % Extends tabular to include paragraph-like column expansion; tabulary also exists (https://ctan.org/pkg/tabularx).
\usepackage{xltabular} % This combines tabularx, longtable, and ltablex.
\usepackage{float} % Makes placing figures and tables easier. In particular, [H] for exact placement (https://ctan.org/pkg/float).
\restylefloat{figure} % If you use the float package, enable this.
\usepackage{wrapfig} % To have figures inline with text.
\usepackage{makecell} % Custom linebreak cells etc.
% \usepackage{rotating} % Rotation of figures and tables (https://ctan.org/pkg/rotating).

% Other packages.
\usepackage{lipsum} % For lorem ipsum placeholder text.
\usepackage[final]{pdfpages} % To include full pdf documents (final forces inclusion even in draft mode).
\usepackage[table]{xcolor} % Custom colors.
\usepackage{csquotes} % Quotes.
\usepackage{menukeys}[os=mac] % To print Apple keys in the appendix. Note that this breaks on Overleaf.
\usepackage{changepage} % For the ability to change page margins in an environment.
\usepackage{soul} % For extra formatting like strikethrough.
\usepackage{siunitx} % Ability to use SI unit symbols (https://ctan.org/pkg/siunitx).



% ****************************** Enumeration Breaking ******************************
% Set boundaries to stop enumations (etc.) from breaking across pages in an ugly way (defaults to 10000).
% \clubpenalty=500
% \widowpenalty=500



% ****************************** References ******************************
% Packages and settings relating to referencing.

% \usepackage{cleveref} % Automatic referencing without explicitly stating fig/table (https://ctan.org/pkg/cleveref).

% Use the document class option 'custombib' to use this section.
\ifuseCustomBib%
  % This loads a custom version of the ieee style that I created (ieeecustom.bbx/cbx). It's the same as ieee, but uses sentence case for references.
  % Without sentence case is "The best paper of all time". With sentence case is "The Best Paper of All Time".
  % `sortcites` is for ordering multiple entries in a single cite by number.
  \RequirePackage[backend=biber, style=ieeecustom, citestyle=ieeecustom, sorting=nty, dashed=false, abbreviate=false, maxnames=99, sortcites=true, backref=true]{biblatex}
  % Add the biblatex reference files. Personal references are separated out into mypubs.bib for convenience.
  \addbibresource{references/references.bib}
  \addbibresource{references/mypubs.bib}

  % Another example. Provide your custom bib package here. In this case, natbib.
  % \RequirePackage[square, sort, numbers, authoryear]{natbib} 
  % If you would like to use biblatex for your reference management, as opposed to the default `natbibpackage` pass the
  %  option `custombib` in the document class. Comment out the previous line to make sure you don't load the natbib package.
  %  Uncomment the following lines and specify the location of references.bib file.
  % \RequirePackage[backend=biber, style=numeric-comp, citestyle=numeric, sorting=nty, natbib=true]{biblatex}
  % \bibliography{references/references} % Location of references.bib only for biblatex.
\fi

% Changes the default 'Bibliography' name into 'References'.
\renewcommand{\bibname}{References}



% ****************************** Custom Commands ******************************
% Custom commands can be defined and configured here.

% For changing the name of ToC, LoF, LoT.
% \renewcommand{\contentsname}{My Table of Contents}
% \renewcommand{\listfigurename}{My List of Figures}
% \renewcommand{\listtablename}{My List of Tables}

% ToC depth and section numbering depth.
\setcounter{secnumdepth}{2}
\setcounter{tocdepth}{2}

% Changes the name of the nomenclature section.
% \renewcommand{\nomname}{Symbols}

% The default value of \appendixtocname and \appendixpagename is 'Appendices'. These commands can change that.
% \renewcommand{\appendixtocname}{List of appendices}
% \renewcommand{\appendixname}{Appndx}

% Custom date format for auto updating use in footnotes for 'as of', such as '[link] as of {\AsOfDate\today}'.
\newdateformat{AsOfDate}{%
\THEDAY\ \monthname[\THEMONTH]\ \THEYEAR%
}
\newcommand{\AsOfToday}{as of {\AsOfDate\today}}



% ****************************** Draft Mode Settings ******************************
% Sets up options for draft mode (enable by using the draft option).

% Uncomment to disable figures in draft mode.
% \setkeys{Gin}{draft=true} % Set draft to false to enable figures in draft mode.

% Set the watermark text.
% \SetDraftText{DRAFT}

% Set the watermark location. Can use 'top' or 'bottom'.
\SetDraftWMPosition{bottom}

% Set the version number for the draft. Defaults to v1.0.
% \SetDraftVersion{v1.1}

% Draft text grayscale value (where 0 is black and 1 is white). Defaults to 0.75.
% \SetDraftGrayScale{0.8}



% ****************************** Todo Notes Settings ******************************
% Uncomment the following lines to enable todonotes.
% \ifsetDraft%
%   \usepackage[colorinlistoftodos]{todonotes}
%   \newcommand{\mynote}[1]{\todo[author=Daniel,size=\small,inline,color=green!40]{#1}}
% \else
%   \newcommand{\mynote}[1]{}
%   \newcommand{\listoftodos}{}
% \fi
% Example todo: \mynote{Enter your note text here.}



% ****************************** Custom Titles ******************************
% titlesec package for configuring the headers. Requires xcolor package.
\usepackage[]{titlesec} % Helpful options: bf, compact, explicit
\definecolor{gray75}{gray}{0.75}
\definecolor{chapterColor}{HTML}{000000}
\definecolor{sectionColor}{HTML}{222222}
\definecolor{subsectionColor}{HTML}{444444}
\definecolor{subsubsectionColor}{HTML}{444444}

% \chapter
% Style 1:
% \newcommand{\hsp}{\hspace{20pt}}
% \titleformat{\chapter}[hang]{\Huge\bfseries}{\thechapter\hsp\textcolor{gray75}{|}\hsp}{0pt}{\Huge\bfseries}
% Style 2:
% \renewcommand{\thechapter}{\Roman{chapter}}
% \titleformat{\chapter}[display]%
%   {\bfseries\Large}%
%   {\filleft\MakeUppercase{\chaptertitlename} \Huge\thechapter}%
%   {4ex}%
%   {\titlerule\vspace{2ex}\filright}[\vspace{2ex}\titlerule]
% Style 3:
\titleformat{\chapter}[display]%
  {\normalfont\Large\filcenter\scshape\color{chapterColor}}%
  {\titlerule[1pt]\vspace{1pt}\titlerule\vspace{0.5pc}\Huge\MakeUppercase{\chaptertitlename} \thechapter}%
  {0.5pc}%
  {\titlerule\vspace{0.5pc}\Huge}

% \section
\titleformat{\section}[hang]%
  {\normalfont\scshape\Large\color{sectionColor}}%
  {\thesection\hspace{10pt}}%
  {0pt}%
  {}

% \subsection
\titleformat{\subsection}[hang]%
  {\normalfont\scshape\large\color{subsectionColor}}%
  {\thesubsection\hspace{5pt}}%
  {0pt}%
  {}

% \subsubsection
\titleformat{\subsubsection}[hang]%
  {\normalfont\normalsize\color{subsubsectionColor}}%
  {\thesubsection\hspace{5pt}}%
  {0pt}%
  {}

% Alternatively: \sffamily or \rmfamily or \scshape can be used.



% ****************************** Code & Syntax Highlighting ******************************
% If you want to include code snippets, then this is a good library.
% An example below defines two languages.

% With help from https://www.overleaf.com/latex/examples/listings-code-style-for-html5-css-html-javascript/htstpdbpnpmt.
\usepackage[final]{listings}
% Novella language syntax.
\definecolor{CodeString}{HTML}{E6DB74}
\definecolor{CodeComment}{HTML}{75715E}
\definecolor{CodeKeyword1}{HTML}{F92672}
\definecolor{CodeKeyword2}{HTML}{195CB3}
\definecolor{CodeIdentifier}{HTML}{272822}
\lstdefinelanguage{Novella}{
  keywords={Variable,Entity,Tag,Group,Sequence,Event,Discoverable,Selector},
  keywords=[2]{Void,Int,Boolean,Double,String,Tangibility,Functionality,Clarity,Delivery},
  keywords=[3]{Type},
  sensitive=true,
  morecomment=[l]{//},
  morecomment=[s]{/*}{*/},
  morestring=[b]',
  morestring=[b]",
  alsoletter={:},
  alsodigit={-}
}
\lstdefinestyle{Novella}{
  % basicstyle={\footnotesize\ttfamily},
  basicstyle={\footnotesize\fontfamily{pcr}}, % Courier (https://www.overleaf.com/learn/latex/Font_typefaces).
  identifierstyle=\color{CodeIdentifier},
  keywordstyle=\color{CodeKeyword1} \bfseries,
  keywordstyle=[2]\color{CodeKeyword2}\bfseries,
  keywordstyle=[3]\bfseries\underbar,
  stringstyle=\color{CodeString}\ttfamily,
  commentstyle=\color{CodeComment}\ttfamily,
  language=Novella
}

% VERY limited HyperTalk syntax.
\definecolor{HTCodeKeyword1}{HTML}{19177C}
\definecolor{HTCodeKeyword2}{HTML}{008000}
\definecolor{HTCodeString}{HTML}{BA2121}
\lstdefinelanguage{HyperTalk}{
  keywords={ask,into,card,field},
  keywords=[2]{put,it},
  morestring=[b]",
  showstringspaces=false
}
\lstdefinestyle{HyperTalk}{
  % basicstyle={\footnotesize\ttfamily},
  basicstyle={\footnotesize\fontfamily{pcr}}, % Courier (https://www.overleaf.com/learn/latex/Font_typefaces).
  identifierstyle=\color{CodeIdentifier},
  keywordstyle=\color{HTCodeKeyword1}\bfseries,
  keywordstyle=[2]\color{HTCodeKeyword2}\bfseries,
  stringstyle=\color{HTCodeString}\ttfamily,
  language=HyperTalk
}
